\documentclass[a4paper]{article}


% I have no idea but the following lines seem necessary to get the header to work wtf?
% \def\nextra
% \def\nisofficial

\def\npart {II}
\def\nterm {Michaelmas}
\def\nyear {2016}
\def\nlecturer {J.\ Miller}
\def\ncourse {Probability and Measure}

\newcommand{\ntoinf}{n \to \infty}
\newcommand{\ntoisr}[1]{\stackrel{\ntoinf}{#1}}

\input{header}

\title{GIRGs thesis - Benjamin Dayan - Marc \& Ulysse }
\author{Benjamin Dayan - bdayan@student.ethz.ch}
\date{\today}

\begin{document}

{\let\clearpage\relax \maketitle}
{\let\clearpage\relax \tableofcontents} 

% \tableofcontents
%\setcounter{section}{-1}
\section{Erdos Renyi Random Graphs}
$G_{n,p}$ has uniform iid edge probability of $p$.

\textbf{Expected Degree} is $(n-1)p$, and $\deg(v) \sim \binomial(n-1, p)$.

If we fix $p = \mu/n$ s.t. $E[\deg(v)] \stackrel{n \to \infty}{\to} \mu$,
then $\deg(v) \stackrel{n \to \infty}{\sim} \Poisson(\mu)$

\subsection{Galton-Watson Branching process}
GWBP with offspring dist. $\D \in \N_0$ generates a random rooted tree $T$ where every vertex independently has 
$\D$ children, e.g. $\D = \Poisson(\mu)$.

\begin{comment}
Question on Thm 2.3: Why cannot we just show that for any given vertex $u$, denoting $C_u$ as its connected
component, that $P(|C_u| = s) \ntoisr{\to} P(|T| = s)$?

Answer: Yes it was immediately in the next lines of the notes :P
\end{comment}
\begin{thm}[2.3 - connected component structure like GWBP trees]
This is a weird statement. We show it precisely by showing that
$P(|C_u| = s) \ntoisr{\to} P(|T| = s)$, for any node $u \in V$.
This we do by coupling the connected component produced (as a BFS)
with the GWBP (only concentrating on outgoing edges (to non-visited nodes)).

Further with any connected component (e.g. of 5 nodes) produced as such,
it has $\to 0$ probability of containing any additional cyclic edges (ignored in the GWBP).
This corroborates e.g. $E[\# \text{triangles}] = \Theta(n^3 p^3)$ is constant.
\end{thm}

Ok

\end{document}