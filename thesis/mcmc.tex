\chapter{MCMC}
\section{Introduction}
The binary classification framework is designed to compare the similarity of two distributions of datapoints. Each graph is a datapoint, and the classifier compares graphs only on higher level features meant to distinguish distributions of graphs, not individual graphs. In particuular these features are node permutation invariant - such as average degree, effective diameter and so on.

An alternative framework is to try to fit a GGM model to a graph so as to compare on an edgewise and likelihood basis. For for the GIRG GGM, this means not just fitting $\hat{\alpha}$ to match the local clustering coefficient, $\hat{c}$ to match the number of edges, and $\hat{\tau}$ to match the degree distribution tail, but to further actually try and infer individual node weights $w_u$, and positions $x_u$. 

We saw this already to some extent with the copy-weight GIRGs - where $\hat{w}_u = d_u$ is fit, as the observed real graph node degrees. The $x_u$ are much harder to fit - for a start any maximum likelihood fit $\hat{x}_u$ would be rotation, reflection, and translation invariant (isometries). Finding any one maximum likelihood fit is impractical for all but the smallest graphs. Instead we try a Markov-Chain MonteCarlo (MCMC) approach to sample a set of locations $\{x_u\}$ from the posterior distribution.

\section{Formulation}
MCMC is a method in the bayesian framework of a parametric generative model. Parameter $\theta$ has prior $p(\theta)$. Datapoint $z \sim p(z | \theta)$. In our case of our node specific fitting GIRG GGM, $\theta = (\alpha, c, \{w_u\}_{u \in V}, \{x_u\}_{u \in V})$, and we focus in particular on the locations $x_u$. $z$ for us is one real graph instance $G$.

The posterior likelihood $p(\theta | z) = \frac{p(z | \theta) p(\theta)}{p(z)}$ is infeasible to compute due to the normalising factor $p(z)$. MCMC instead uses the non normalised $Q(\theta) = p(z | \theta) p(\theta)$ which can be evaluated, to set up a Markov Chain with states $\theta \in \Theta$, and transition probabilities derived from $Q(\theta)$. In particular we will use a Metropolis-Hastings style markov chain. With a proper MC state transition probability $p(\theta \to \theta')$, we can perform a random walk on the MC state space that converges in the limit to the posterior distribution $\theta \sim p(\theta | z)$. If the jth step of the random walk yields $\theta^j$, given sufficient burn in and spacing, we can sample $\theta^{j_1}, \theta^{j_2}, ...$ from the posterior. We will be content with just one posterior sampled $\hat{\theta}$ (NB not a maximum likelihood estimator), and use this to evaluate our overall GIRG model fit to the real graph.

I think we do a Gibb's sampling approach. This means both breaking down $\theta$ into subcomponents $\theta_i$, randomly choosing one to propose a new state for, i.e. $\theta' = (\theta'_i, \theta_{-i})$, and using the $Q_i(\theta_i)$ instead of $Q(\theta)$ - i.e. just the marginal non-normalised posterior. In our case $Q(\theta) = p(G | \theta) p(\theta)$, so the uniform prior $p(\theta)$ can be dropped everywhere as $p(\theta) = 1\; \forall \theta$, and then $p(G | \theta) = \prod{u, v \in V; u \neq v} p(e(u,v) | w_u, w_v, x_u, x_v, \alpha, c)$. This lends well to Gibb's sampling - we need concentrate only on $Q_u(\theta_u) = \prod{v \neq u} p(e(u,v) | ...)$.

Key elements of the MCMC approach are 1. burn in time, and 2. proposal distribution. 

Burn in time can be quite long. One key component is a good initialisition. With the GIRG model, the natural initialisation for $x_u$ would be to follow the prior $x_u \sim U([0, 1]^d)$. Instead we use an initialisation based on d-truncated diffusion map from the real graph connectivity. We also try to optimise our code and use multiprocessing to speed up the random walk.

The proposal distribution should be designed to maximise chances of acceptance. It seemed reasonable to stochastically propose either a small local perturbation, or a random jump to anywhere in the cube, with some probability of either (we elected for 70\% small perturbation). A random uniform jump is useful to try and find a completely new location for $x_u$ that could suit (hopefully near to its neighbours) - in fact another good proposal would be to randomly choose a neighbour $v \sim u$, and move $x_u$ to a random offset of $x_v$. A small perturbation $x_u' = x_u + \epsilon$ is good as assuming $x_u$ has high likelihood, somewhere nearby might have even higher.


\section{Bayes Factor likelihood comparisons}
\begin{enumerate}
    \item Using one MCMC posterior sample of $\{x_u\}_{u \in V}$ and maximum likelihood estimates $\hat{\alpha}, \hat{c}, \hat{w}_u$, we estimate the likelihood $p(\cG_{\GIRG} | G)$, and can perform a comaprison $p(\cG_{\GIRG-1d} | G)$ vs $p(\cG_{\GIRG}-2d | G)$ vs $p(\cG_{CL} | G)$ etc.
    \item Results show that $p(\cG_{\GIRG}-1d | G)$ is better than 2D, 3D GIRGs, but not as good as Chunglu
    \item Results show that simply introducing a failure rate of $0.5$ onto the GIRGs (unfortunately not correcting for LCC), the model probabiltiy estimate for 1D GIRG then beats out chunglu (all copyweights)
\end{enumerate}

