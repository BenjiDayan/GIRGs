\chapter{Introduction}
\label{chap:introduction}
% \minitoc
\section{Thesis Introduction}
Graphs are a useful model for describing networks of interconnected entities as an abstraction of real-life systems. Real-world graph data are abundant, and many graphs exhibit similar structural characteristics. Several graph-generating models have been proposed, each justified by its reasonable formative mechanisms which create sets of graphs similar to those observed in the wild. Studying these generative models theoretically provides insights into real networks and is useful for creating synthetic graphs for testing purposes.

\textit{Geometric Inhomogeneous Random Graphs (GIRGs)} are one such graph-generating model, first proposed in \cite{bringmann2016average} as a generalisation of \textit{Hyperbolic Random Graphs}. GIRGs allow significant flexibility in the geometric component of their generative structure, thereby enabling better modeling of a broader range of real graphs. Their definition encapsulates the intuitive notion present in many graphs that nodes are more likely to be connected if they are \q{close together}. Furthermore, GIRGs exhibit inhomogeneity between nodes, resulting in a power law degree distribution, a characteristic observed in many real graphs.

This thesis evaluates the ability of GIRGs to simulate a dataset of $104$ social network graphs obtained from the \href{https://networkrepository.com/}{Network Repository} (\cite{rossi2015network}, \url{https://networkrepository.com/}). These graphs represent Facebook social networks from various towns in the US, thus exhibiting structural similarities, and range from $n=1000$ to $n=100,000$ nodes. In \cref{chap:GGM}, we assess the mostly unmodified GIRG prior distribution of graphs based on their high-level similarities to the real graphs. In \cref{chap:likelihood_point_estimation}, we perform a more intensive posterior style fit of GIRG to replicate real graphs at the node/edge level.

In \cref{chap:GGM}, the first task of assessing GIRG prior realism is conducted using a high-level feature dataset classification comparison framework proposed in \cite{blasius2018towards}. Our results provide evidence that GIRGs are more realistic candidates than simpler generative models with respect to geometric features such as node closeness centrality, betweenness centrality, local clustering coefficient, and graph effective diameter.
We compare the realism of different GIRG variants on these features and find that while max norm torus GIRGs provide a good baseline, cube GIRGs perform better on graph effective diameter, and mixed/minimum component distance GIRGs excel in closeness centrality.
In most (but not all) cases, we find that a small geometric dimensionality of $d=1$ already provides near-peak GIRG realism, with little benefit from higher-dimensional GIRGs.

Additionally, in \cref{chap:graph_kernels}, we attempt to apply the method of \textit{graph kernels} as an alternative framework for model selection between the GIRG prior model and other generative graph models with respect to our real graph dataset. Unfortunately, we were unable to produce reasonable results.

% and with respect to which statistics that GIRGs are most realistic. Only on some statistics do additional geometric dimensions and varying the type of GIRG make much difference - most results don't differ much from the basic 1-dimensional GIRG.

% We additionally explore the use of Graph Kernels as an alternative framework to explicit sets of high level graph statistics, but fail to produce reasonable results.

In the second task of GIRG posterior fitting, we aim to infer the node geometric locations of a given posited GIRG generative model for an input real graph. In \cref{chap:diff_maps}, we explore the method of diffusion maps as a computationally fast initial estimate of locations with good internode distances, and as a means of proposing the underlying geometric dimensionality. In \cref{chap:likelihood_point_estimation}, we describe methods to further refine node locations using either a markov chain Monte Carlo or maximum likelihood-based fitting approach. The full fitting procedure exhibits decent performance in replicating real graphs and supports the hypothesis that one dimension is mostly sufficient for this dataset of social networks.

To support all objectives, we also cover the different types of GIRGs and their unique , and demonstrate practical methods for sampling and working with them.



\chapter{GIRGs - Variants and Generation}
\minitoc
\section{GIRGs definition}
\label{sec:GIRG_def}
In this section, we define GIRGs and identify their key features.

As per \cite{bringmann2019geometric}, the GIRG model is a random graph model characterized by the edge connection probabilities given by the formula:

% TODO below (div by n, infty norm) is what I had before, but in fact bringmann2019geometric uses below with div by W and ambiguous norm.
% \begin{equation}
%     p_{uv} = \Theta \left ( \min \left \{
%         1,
%         c \left (
%             \frac{w_u w_v / n}{ \norm{x_u - x_v}_\infty^d}
%         \right )^\alpha
%     \right \}
%     \right )
% \end{equation}

\begin{equation}
    p_{uv} = \Theta \left ( \min \left \{
        1,
        \left (
            \frac{w_u w_v / W}{\norm{x_u - x_v}^d}
        \right )^\alpha
    \right \}
    \right )
\end{equation}

Here, $(w_u)_{u \in V}$ are node-specific weights in $\R^+$, and $(x_u \in \chi)_{u \in V}$ denote positions of nodes in a particular geometric space $\chi$, often taken to be the d-dimensional torus
\footnote{The d-dimensional torus also resembles $[0, 1]^d$, but opposite faces of the cube are identified. Picture Pac-Man, or donuts. Just not Pac-Man eating a donut...}
$\T^n$, or the d-dimensional unit cube $[0, 1]^d$.
The normalising factor $W^{-1}$, where $W = \sum_{u \in V} w_u$, ensures that the expected degree of a node $u$ remains constant even as $n \to \infty$; although $u$ will have more possible other nodes to connect with, this increase is counterbalanced by the declining probability of connecting to any one of them.


% This edge probability has a geometric factor from the distance $r_{uv} = \norm{x_u - x_v}_\infty$, which is taken to the dth power so that the number of edges is consistent across different values of $d$.
% This inversely scales the probabilities so that nearby points (small $r_{uv}$) are more likely to have an edge than those further apart.
% The $W^{-1}$ normalising factor ensures that for a node $u$, as $n$ increases (even $n \to \infty$), though it will have more possible other nodes to connect to, its expected degree won't change.

% Writing $\rho_{uv} = \left ( \frac{w_u w_v / W}{\norm{x_u - x_v}^d} \right )$ to save on writing, this is like the fundamental "score" of edge probability, which must be modulated to become the final probability (e.g. making sure its bounded above by 1).
% In particular the exponent $\alpha \in (1, \infty]$


\subsection{Geometry}
The geometric space $\chi$, and the distance function $\norm{\cdot}$ (not necessarily a norm) can vary a great deal without affecting key properties of GIRGs.

The edge probability $p_{uv}$ has a geometric factor from the distance $r_{uv} = r(\vec{x}_u, \vec{x}_v) = \norm{x_u - x_v}$.
This inversely scales the probabilities so that nearby points (small $r_{uv}$) are more likely to have an edge than those further apart.
Depending on the space $\chi$ and the function $r$ chosen, it may be a norm, a metric, or not even.
$r_{uv}$ is taken to the dth power so that the number of edges is consistent across different values of $d$.
Such proximity influenced $p_{uv}$ is one way of bringing about the phenomenon of \textit{clustering}, common in many real life graphs, whereby subgroups of nodes might have more edges within themselves than expected by chance.



% The geometric space $\chi$, and the distance function $\norm{\cdot}$ (not necessarily a norm) can vary a great deal without affecting the key properties of GIRGs. $\chi = \T^d$ the d-dim torus is very handy for proofs, as the viewpoint of any node $x_u$ is equivalently at the "centre" of the space. For real applications the d-dim cube $\chi = [0, 1]^d$ can be more realistic; however it is no longer symmetric; nodes with locations near the edge of the cube are different from those near the centre.

Taking the distance function as the max norm, i.e. $\norm{\vec{x}} = \norm{\vec{x}}_\infty = \max_i |x_i|$ is also nice for proofs, but can be replaced by euclidean or other norms - by the equivalence of norms there isn't much difference. Note on the torus the max norm isn't strictly a norm, rather we abuse notation to write:
\begin{align}
    ||\vec{x} - \vec{y}||_\infty &= \max_i |x_i - y_i|_C\\
     |x_i - y_i|_C :&= \min(|x_i - y_i|, 1 - |x_i - y_i|)
\end{align}
The \textit{minimum component distance} (MCD) defined as $\norm{x}_{\mathrm{mcd}} = \min_i |x_i|$ doesn't even satisfy the triangle inequality, but still retains most of the GIRG properties. This can make sense if we want two nodes to have a higher edge probability if they are close in at least one dimension, rather than every dimension.

$\alpha \in (1, \infty]$ is another parameter affecting the geometry. The edge probability formular distinguishes \textit{short edges} where $r_{uv} < (w_u w_v / W)^{1/d}$ and $p_{uv} = \Theta(1)$ (for these the $\min(\cdot, 1)$ takes affect to ensure $p_{uv} \leq 1$), and \textit{long edges} where $r_{uv} > (w_u w_v / W)^{1/d}$, and $p_{uv}$ decays to zero with increasing distance. Larger values of $\alpha$ speed this decay, so more strongly restrict the number of longer distance edges. $\alpha=\infty$ enforces a sharp cutoff to $p_{uv} = 0$ for long edges.

% The reason for the power of $d$ in $r_{uv}^d$ is to keep the edge probabilties equivalent with respect to the volume of space in different dimensions.
% The more general formula replaces replaces $||x_u - x_v||^d = r_{uv}^d$ with $Vol(B_r)$ the volume of the ball of radius $r=r_{uv}$.
% For norms, $Vol(B_r) = \Theta(r^d)$, e.g. in the $\infty$-norm, $Vol(B_r) = (2r)^d$ as a cube with side-length $2r$.
% The euclidean ball has some $\pi$'s in the formula. For the minimum component distance, $Vol(B_r) = \Theta(r)$.
% Volumetric equivalence makes sense in that the edge probability $p_{uv} | x_u, w_u, w_v$ in the Torus is determined by integrating $\int_{r=0}^{r=1} p_{uv}(r) p(r) dr = \int_{Vol=0}^{Vol=1} p_{uv}(Vol) dVol$, where $Vol(r) = Vol(B_r)$ is the volume of the ball of radius $r$.
% Hence across different GIRGs of different dimensions $d$, keeping $p_{uv}(Vol)$ the same function therefore keeps the pairwise edge probabilities $p_{uv}$ the same (but not the joint edge probability distribution of course).

\subsection{Similarity to Chung-Lu} 
A key property of the GIRG model is that it's a geometic variant of the Chung-Lu model. The Chung-Lu model decrees edge probabilities between nodes for a given node weight sequence $(w_u)_{u \in V}$ as $p_{uv} = \Theta(\min\{1, \frac{w_u w_v}{W}\})$.


In a GIRG, geometry influences edge proabilities, however when you consider any two nodes $u,v$, marginalising over all of their possible locations (whether they're close by or far apart), you get $E_{x_v}[p_{uv} | x_u, w_u, w_v] = \Theta(\min\{1, \frac{w_u w_v}{W} \})$, precisely like Chung-Lu just forgetting node locations $(\vec{x}_u)_{u \in V}$.

Therefore properties/results about Chung-Lu generated graphs carry through automatically to GIRGs, such as the fact that $E[d_u] = \Theta(w_u)$ - the expected degree of a node $u$ with weight $w_u$ is proportional to $w_u$.

This seems strange as we have the $(w_u w_v / W)^\alpha$ term in the edge probabilities, however as introduced above, $\alpha$ only affects the decay rate of edge probabilities for long edges, which is only a constant fraction of total edge probability, as long as $\alpha > 1$. If $\alpha$ were smaller we would actually blow up the degree $d_u$ far beyond $w_u$, as the large number of potential long edges would have too many realised as actual edges, and dominate the short edges.

% For this to hold we actually need to have $\alpha > 1$. - otherwise there are too many long distance edges. Essentially $\alpha > 1$ depresses the number of long distance edges to be $\Theta(\min\{1, \frac{w_u w_v}{W} \})$. No matter how large $\alpha$ is, there will always be $\Theta(\min\{1, \frac{w_u w_v}{W} \})$ short distance edges: $u \sim v$ where $p_{uv} = 1$.


\subsection{Power Law Degree Sequence} The weight sequence of a GIRG is usually assumed to be generated from a power law distribution with an exponent $\tau \in (2, 3)$. This can also be loosened to just include the tail (larger) weights, for instance just the top $10\%$ largest weight nodes. This allows the GIRG to match the common real network phenomenon of a power law tail degree sequence, since node expected degrees in the GIRG are proportional to their weight.
%  (at least within some tolerance and in the large weight tail).
\begin{align*}
\eqname{exact power law distribution}
& \mathrm{w} \sim \powerlaw(\tau)
\\
\eqname{pdf (default $x_{\min } = 1$)}
& p(w) \propto w^{-\tau} \text{ for } w \in [x_{\min }, \infty]
\\
& p(\mathrm{w} \geq w) \propto w^{1 - \tau}
\end{align*}
% A general In practice a power law degree sequence is required to have a looser version of $\# \{w_u | w_u \geq w\} = \Theta(w^{1 - \tau})$.
The power law distribution is heavy tailed (heavier than exponentially decaying tails). $\tau > 2$ is important to ensure that $E[\mathrm{w}] = \Theta(1)$, which means that although we may have some very large $w_i$ in a sequence $w_1, ..., w_n$, still the majority of the total weight is in smaller valued $w_i$'s. 


\subsection{Volume generalisation}
The volume formulation of a GIRG presented in \cite{bringmann2019geometric} provides a consistent way to handle different geometries:
\begin{equation}
    p_{uv}(r) = \min \left \{ 
        1,
        c \left (
            \frac{w_u w_v / W}{Vol(r_{uv})}
        \right )^\alpha    
    \right \}; \quad \rho_{uv} = \frac{w_u w_v / W}{Vol(r_{uv})}
\end{equation}
Here $Vol(r_{uv}) = Vol(B_{r_{uv}})$ is the volume of the ball of radius $r$ using the distance function $r(x_u, x_v) = r_{uv}$, which should be symmetric.
For example with the max norm, $Vol(r_{uv}) = (2r_{uv})^d$ as a cube with side-length $2r_{uv}$.

Having volume in the edge probability formula is actually a generalisation of taking $r_{uv}$ to the dth power - the point being to make $p(u \sim v | x_u, w_u, w_v) = E_r[p_{uv}(r)] = \Theta(w_u w_v/n)$, regardless of dimension. We'll actually derive in \cref{subsec:average_degree_formula} a precise formula for expected edge probabilities / node degrees that holds across all different types of volume formulation torus GIRGs. This will however need an exact GIRG edge probability formula, without the outer $\Theta ( \cdot )$, which we present in \cref{sec:blasius_cpp_generation}.

% \section{Practical Realisation of Different GIRGs}
% \section{Fitting GIRGs for Blasius evaluation framework}
% As explained in \cref{chap:GGM}, we wish to compare GIRGs with other GGMs for their ability to fit a dataset of real social network graphs. 
% In \cref{sec:fitting_GGM}, we introduced the ABC method for fitting a GGM to a particular real graph instance $G = (V, E)$, which we will use for GIRGs, similarly to how \cite{blasius2018towards} fits Hyperbolic Random Graphs.

% A first important step for ABC is the ability to generate GIRGs, i.e. sample $G \sim \cG_{\GIRG}(\theta)$.



\section{Blasius C++ GIRG generation}
\label{sec:blasius_cpp_generation}
\cite{blasius2022efficiently} provides a C++ implementation of an efficient algorithm for sampling a max torus GIRG.
% We started by using the C++ implementation of GIRG generation in \cite{blasius2022efficiently}. 
Their GIRG formulation uses a toroidal geometry $\chi = \T^d$, with edge probabilities 

\begin{equation}
    p_{uv} = \min \left \{ 
        1,
        c \left (
            \frac{w_u w_v / W}{\norm{x_u - x_v}_\infty^d}
        \right )^\alpha    
    \right \}
    \label{eq:blasius_GIRG_puv}
\end{equation}
i.e. no outer $\Theta$ which gave us more flexibility in the general GIRG definition - just one fixed inner constant $c$ in order to be concrete. 
Note that this still falls under the wider GIRG definition 
% of $p_{uv} = \Theta \left [ 
%     \min \left \{ 
%         % 1,
%         % \left (
%         %     \frac{w_u w_v / W}{\norm{x_u - x_v}_\infty^d}
%         % \right )^\alpha    
%         ...
%     \right \}
% \right ]$,
as $p_{uv}$ always lies in the interval 
\begin{equation}
    c \min \left \{ 
        1,
        \left (
            \frac{w_u w_v / W}{\norm{x_u - x_v}_\infty^d}
        \right )^\alpha    
    \right \} \leq p_{uv} \leq \min \left \{ 
        1,
        \left (
            \frac{w_u w_v / W}{\norm{x_u - x_v}_\infty^d}
        \right )^\alpha    
    \right \}
\end{equation}
if $c \leq 1$, and with the upper and lower bounds swapped for $c > 1$.


They implement the algorithm proposed by \cite{bringmann2019geometric}, which has expected runtime $O(n)$ i.e. linear in the number of nodes, for a GIRG with a power law generated weight sequence.
This is optimal: such a GIRG has expected number of edges $\Theta(n)$, due to the fact that nodes have finite expected weight $E[w_u] = \Theta(1)$, and expected degree proportional to their weight. I.e. just to write down / store the set of edges $E = \{(u, v) | u \sim v \}$ we need $O(n)$ time/space.

\subsection{GIRG Generation with Python}
Unfortunately this GIRG sampling algorithm scales exponentially in dimension $d$, as it involves breaking the Torus space down hierarchically into smaller cubes and considering edges between neighbouring cubes. We struggled to use it in experiments for $d \geq 4$.
% has degree proportional to weights $E[d_u] \propto E[w_u] = \Theta(1)$, so with high probability has $\Theta(n)$ number of edges, i.e. linear in number of nodes.
% Hence a linear runtime algorithm ($O(n)$) is optimal, as even an oracle must write out the list of edges sequentially in $\Theta(n)$ time.
Hence we also implemented our own GIRG generation code in python for ease of modification to deal with different types of GIRG and to cope with higher dimensions $d$. 
% We use the same specific $p_{uv} = \min(1, c \cdot)$ as in \cref{eq:blasius_GIRG_puv}. 
Our algorithm is the most basic $O(n^2)$ checking of every single node pair $(u, v)$, and also has $O(n^2)$ memory requirement. The basic steps are:
\begin{enumerate}
    \item sample $O(n)$ node weights $w_u$ from a power law distribution
    \item sample $O(n)$ node locations $x_u$ from a uniform distribution on the torus
    \item compute the $O(n^2)$ pairwise node distances and hence edge probabilities
    \item for each of the $O(n^2)$ potential edges, sample a Bernoulli random variable with the edge probability as its parameter, to determine whether the edge exists
\end{enumerate}
% Unfortunately the Blasius algorithm scales quite badly in dimension $d$, such that for larger graphs of with $d \geq 4$, the python implementation was preferrable.




% to sample node weights and locations ($O(n)$); compute the ${n \choose 2}$ pairwise edge distances and hence edge probabilities, each of which is used to sample a final Bernoulli random variable to determine whether the edge exists or not ($O(n^2)$).



% We use this code so much so will need to take note of their preferred notation:
% \begin{equation}
%     p_{uv} = \min \left \{ 
%         1,
%         c \left (
%             \frac{w_u w_v / W}{\norm{x_u - x_v}_\infty^d}
%         \right )^\alpha    
%     \right \}
% \end{equation}
% Where they take $x_u \in \chi = \T_d$ the torus. They use a variant of the GIRG where normalisation by $n$ is replaced by that of $W = \sum_{u \in V} w_u$. For $w_u$ obeying a $\tau > 2$ power law distribution, $W = \Theta(n)$ with high probability, so this is fine.


% I.e. the wider GIRG definition just requires that there are some lower and upper bounding constants $c_L, c_U$ such that for every pair of nodes $u, v$, their edge probabilities are given as $c_L \min \{ 1, (...)^\alpha \} \leq p_{uv} \leq c_U \min \{ 1, (...)^\alpha \}$. If these bounds are fixed for increasing $n$, then all the nice properties of GIRGs can be proven!


% Our codebase is set up to match \cite{blasius2022efficiently}, however we may use alternative notation in this document as the use case calls for it.


% Therefore the most canonical and tight of GIRGs would be defined as having edge probabilities all precisely as $p_{uv} = 
%     \min \left \{ 
%         1,
%         \left (
%             \frac{w_u w_v / W}{ (Vol(\norm{x_u - x_v}) / Vol(\mathbb{T}))}
%         \right )^\alpha    
%     \right \}
% $, where $Vol(\mathbb{T}) = 1$ for the d-dimensional unit Torus $\mathbb{T} = [0, 1]^d$.




% Neither quite matches the exact volume formulation: For MAX GIRGs, $Vol(r) = (2r)^d$ not $r^d$. Meanwhile e.g. for Euclidean GIRGs there's probably something more like $\pi r^d$ idk sphere volume formulae.

% There's some weird conversion formula:
% $x_i \sim [0, n^{1/d}]^d$, $\tilde{x}_i \sim [0, 1]^d$, $\hat{x}_i \sim \bar{w}^{1/d} [0, n^{1/d}]^d$. This means that $\hat{r}^d = \bar{w} r^d = \bar{w} n \tilde{r}^d = W \tilde{r}^d$.

% My original implementation used $r$, which is nice as "space is real", however $\hat{r}$ might be better.


% \subsubsection{equivalence of C++ GIRG definition and Johannes definition}
% i.e. c min(1, ()**alpha) vs min(1, ()**alpha) - I think they're not
% equivalent, but they are up to Theta(min(1, ()**alpha)) which is What we wanted.

% also maybe of their weird weights definition.

% \paragraph{ABC recap}
% Hence our full power law weighted torus GIRG parametrisation is $\theta = (n, d, c, \alpha, \tau)$

% We fit the model to a certain real graph instance $G = (V,E)$ in a few steps, using ABC as introduced in \cref{sec:fitting_GGM}, and different distance metrics $d(G, G')$ for each subcomponent of $\theta$. All we need to do is be able to propose values of $\theta$ and generate a graph $G' \sim \cG_{\GIRG}(\theta)$ from them.

% Our method used is a form of Approximate Bayesian Computation (ABC). In short, ABC fits a parametric bayesian model to data $D$ without computing a posterior likelihood $p(\theta | D)$ (infeasible), rather by sampling $\theta$ from the prior, and accepting $\theta$ if the simulated data $D'$ from $\theta$ is "close enough" to $D$, under some distance metric $\rho(D, D')$.

% In our case we fit $\theta$ given just one datapoint $D = G$, and we use different distance metrics $\rho$ for fitting each part of $\theta = (n, d, c, \alpha, \tau)$




\section{Cube GIRGs}

Cube GIRGs are an alternative formulation to Toroidal GIRGs. Instead of the $d$-dimensional torus $\chi = \mathbb{T}^d$, we have the $d$-dimensional cube $\chi = [0,1]^d$. Thus the tradeoff is to lose the symmetry and simplicity of the torus, for the ability to hopefully generate more realistic graphs, as few situations in the real world are Toroidal.

% For example with our socfb social network graphs, likely geometric quantities are home address (as a 2d vector location - these networks are from a small town/city/university - only on planet earth scale do geographic locations become somewhat more toroidal), political leaning (typically a 2d axis of economic left/right, and authoritarian/libertarian - e.g. an extreme authoritarian is unlikely to get along with an extreme libertarian), athletic proclivity etc. all generally are non toroidal. Perhaps one toroidal-esque quantity could be that in all cases, a pair of positive and negataive far out leaning people share in common their marginalisation / non conventionality - this could potentialily be captured on a 1D non toroidal normie vs hipster scale.  

For example with our socfb social network graphs, likely geometric features for each individual are home address, age, athletic proclivity, political leaning etc.
In the environ of a town/city, home address is a 2D vector
\footnote{Only on the whole planet scale would home address look more torus like - strictly speaking a 3d sphere, or perhaps a cylinder if you assume that nobody lives near to the North/South Pole}.
Age and athleticisim are clearly both non toroidal (and hence cube like) quantities.
Political leaning could be e.g. a 2d axis of economic left/right and authoritarian/libertarian. This might bear a toroidal-esque characterisation whereby two individuals on far opposite extremes of an axis share some common ground, e.g. a hardcore communist and a fascist might both share values such as railing against the status quo, or enjoying political rallies.
No model is perfect; this could potentially be captured as an extra (non toroidal) feature dimension placing individuals on a scale of normie to hipster.

\subsection{Cube GIRG Formulation}
The neat correspondence between volume and distance unfortunately breaks down when translating from torus GIRGs to cube GIRGs, due to the loss of spatial symmetry.
% Where before with the max norm we could write $Vol(r_{uv}) = (2 \norm{x_u - x_v}_\infty)^d$, now in cube geometry $Vol(B_{r_{uv}}(u)) = Vol(B_{r_{uv}}(u)) \cap [0, 1]^d$
With the max norm,
\begin{align}
    \text{in the torus} \quad Vol(r_{uv}) &= (2 \norm{x_u - x_v}_\infty)^d
\\
\text{now in the cube} \quad  Vol(B_{r_{uv}}(u)) &= Vol(B_{r_{uv}}(u)) \cap [0, 1]^d
\end{align}
i.e. volume only counts within the cube itself, and hence oftentimes $Vol(B_{r_{uv}}(u)) \neq Vol(B_{r_{uv}}(v))$. 

We will use a simpler cube GIRG formulation based solely on distance, i.e. the factor of $r_{uv}^d$ in $p_{uv}$. This is conceptually easy to understand, and allows for a neat coupling with the original Torus geometry whereby $r_{uv}^\text{cube} \geq r_{uv}^\T$, with most pairs having $r_{uv}^\text{cube} = r_{uv}^\T$. We can say that a GIRG in cube geometry is stochastically dominated by its torus geometry counterpart as $p_{uv}^\text{cube} \leq p_{uv}^\T$, and is hence strictly sparser.

A volume based formulation of edge probabilities is more complicated - it could look something like replacing $Vol(r_{uv}) \mapsto \sqrt{Vol(B_{r_{uv}}(u)) Vol(B_{r_{uv}}(v))}$.
Intuitively in the social network analogy, this is like saying that all people, no matter how extreme their geometric location (near the edge of the cube), have the same desire to make friends (modulo their inhomogeneous weights as that is the literal introversion / extroversion factor in the GIRG model). 
If instead of having undirected edges $u \sim v$ we had directed edges $u \to v$, we could even have a model where $p(u \to v) \propto \sqrt{Vol(B_{r_{uv}}(u))}$, such that $u$ would in expectation have the same number of outgoing edges whatever its location $\vec{x}_u$.
% Thus if we were modelling directed edges, we could have equality in $E_r[p_{u \to v}(r)]$ in all GIRGs of differing distance metric, number of dimensions and torus / cube geometry by using volume based formulations.
Unfortunately desire to make friends doesn't equate to actual friends, as they have to want you back.
So if a node $u$ is near the edge of the cube, while it would send out the normal volume based amount of friendship invitations ($u \to v$), as most of these go to nodes $v$ nearer the centre of the cube, fewer will reciprocate ($v \to u$). In the directed torus geometry analogy, due to symmetry, every $p(u \to v) = p(v \to u)$. This means more pairs of $u \leftrightarrow v$, and fewer pairs of \q{unrequited friendships} i.e. $u \to v;\, v \not\to u$.
% In the torus geometry (all weights being equal), every $u$ would have the same expected number of outgoing edges as incoming.
% In the torus geometry version of this analogy, due to symmetry, every $p(u \to v) = p(v \to u)$. This would actually end up with more pairs of $u \leftrightarrow v$.
Hence a volume based cube GIRG in the undirected setting would still have higher average degree than the distance based version, but lower than the torus.

% however if an undirected edge $u \sim v$ requires both $u \to v$ and $u \gets v$, so even if $u$ has many outgoing friendship invitations, if all its proposed friends have many friendship options ($u$ is the mountain cave hermit soliciting friends in the nearby village), $u$ will still end up with fewer friends than average, however more than in the distance based formulation:

% We will use a simpler cube GIRG formulation based solely on distance, i.e. the factor of $r_{uv}^d$ in $p_{uv}$. This is conceptually easy to understand, and allows for a neat coupling with the original Torus geometry whereby $r_{uv}^C \geq r_{uv}^\T$, with most pairs having $r_{uv}^C = r_{uv}^\T$. We can say that a GIRG in cube geometry is stochastically dominated by its torus geometry counterpart (and hence strictly sparser). 


% $r_{uv}^C$ vs $r_{uv}^\T$ in a cube vs torus satisfies $r_{uv}^C \geq r_{uv}^\T$. For the $\infty$ norm, $r_{uv}^C = r_{uv}^\T \iff r_{uv}^C \leq \frac{1}{2}$. The simplest edge probability translation therefore is just to use the same $r_{uv}^{-d \alpha}$ factor, such that Cube GIRGs are just a more sparse version of Toroidal GIRGs. We call this the distance based formulation. 

% An alternative volume based formulation, seeking to replicate the $r_{uv}^d \propto Vol(B_{r_{uv}})$, is complicated by the lack of symmetry. The whole point of the general volume formulation of edge probability was that the node $u$ has a set edge probability to neighbours $v$ at a certain volumetric distance - where now in the cube case, we could take $Vol(B_{r_{uv}}(u)) = Vol(B_{r_{uv}}(u)) \cap [0, 1]^d$, i.e. volume only counts within the cube itself.
% This is not symmetric in that $Vol(B_{r_{uv}}(u)) \neq Vol(B_{r_{uv}}(v))$: if $u$ is closer to the edge of the cube and $v$ closer to the centre, then $Vol_u < Vol_v$.
% However dealing with undirected edges, we must have one agreed upon formula for $p_{uv}(r)$, and so a natural formula would be to replace $Vol(B_{r_{uv}}) \mapsto \sqrt{Vol(B_{r_{uv}}(u)) Vol(V_{r_{uv}}(v))}$. This would basically seek to compensate a $u$ near the edge, allowing it to better seek far away neighbours.
% Even still, this does not end up with all nodes of a weight having the same expected number of neighbours regardless of their location. Rather it's like we had directed eges, yielding undirected edges if both directions are present: $u \to v$ and $v \to u$ implies $u \sim v$. With this volume based formulation, every node has out edges based on its own weight and volumetric distance to other nodes. $p(u \sim v) = p(u \to v) p(v \to u)$. If we ignored the minimum term in the edge probabilities (i.e. saying that $p(u \to v) \propto w_u / \sqrt{Vol(B_{r_{uv}}(u))}$, not upper bounded by $1$, then actually all nodes would have the same expected number of neighbours, regardless of their location.

% In the social network analogy, the distance based formulation is like saying that people make friends only with those who live close by.
% Thus if a person lives like a hermit in a mountain cave far from the neaerest village, they're unlikely to have many friends.
% The volume based formulation is saying that all people, no matter how extreme their geometric location, have the same desire to make friends (holding their inhomogeneous weights constant, as that is the literal introversion / extroversion factor in the GIRG model). So even if you're the most far left communist, you will still make as much total social effort, primarily on far lefters, but also towards even some centre left potential friends. However to most central leftists, the communist is too extreme and won't get much social budget compared to the large choice of more politically nearby folks.

% TODO think more if good idea?
% Another possible model for spheres of influence about nodes near the edge of the cube is one where spheres of influence are no longer radially symmetric. In the volume based formulation in a 2D space, where $u$ is near the north edge of the square, efforts to find friends in the north are redirected equally along the east, south and west at the same radius. In a physical analogy, it's possible instead that "feelers" to the north are redirected along the path of least resistance, to the east and west. In a political spectrum analogy, more aligned to the min GIRG philosophy, if a person is located in the far social left of the cube, but economically central, they may seek friendships amongst those who are either similarly economically or similarly socially. Hence if there are no people more socially left, they may redirect their social dimension friendship effort budget to those a bit more socially right of themselves, rather than also to those economically left/right close by. We don't pursue this model!

\subsection{Cube GIRG generation - coupling algorithm}
We use the distance based cube GIRG formulation, which permits a simple generation algorithm that couples to a generated torus GIRG. All that is needed are the original weights and point locations, and the edges present in the torus GIRG. The pseudocode is shown in \cref{alg:cube_coupling}. This runs in $O(n)$ extra time on top of the initial torus GIRG generation.

The coupling works for any pair of torus/cube distance functions $r_{uv}^\text{cube} \geq r_{uv}^T$, which results in stochastically dominated edge probabilities $p_{uv}^\text{cube} \leq p_{uv}^\T$. Thus we can couple the events $u \sim v$ in the cube GIRG as a subset of the events $u \sim v$ in the torus GIRG. We just have to inspect each of the $\Theta(n)$ edges in the torus GIRG, and flip a $\Bernoulli(p_{uv}^\text{cube} / p_{uv}^\T)$ coin for each to decide whether to keep / delete the edge for the cube GIRG.

\begin{algorithm}
    \caption{Generate cube GIRG from torus GIRG via coupling}\label{alg:cube_coupling}
    \begin{algorithmic}
    \Require $n$, $d$, $c$, $\tau$, $\alpha$
    \State $\left ( G, \{x_u\}_{u \in V}, \{w_u\}_{u \in V} \right ) \gets \text{torus-}\GIRG(n,d,c,\tau, \alpha)$
    \For{$(u,v) \in E(G)$}
        \State $p_{uv}^\T = \min \{1, c \left (
            \frac{w_u w_v / W}{(r_{uv}^\T)^d} \right )^\alpha \}$
        \State $p_{uv}^\text{cube} = \min \{1, c \left (
            \frac{w_u w_v / W}{(r_{uv}^\text{cube})^d} \right )^\alpha \}$
        % \If{$p_{uv}^C < p_{uv}^\T$}
        % \State $p_{\text{del}} \gets 1 - \frac{p_{uv}^C}{p_{uv}^\T}$
        \State $p \gets U[0,1]$
        \If{$p > \frac{p_{uv}^\text{cube}}{p_{uv}^\T}$}
            \State delete edge $(u,v)$ from $G$
        \EndIf
        % \EndIf
    \EndFor
    \State \Return $G$
\end{algorithmic}
\end{algorithm}




\section{Minimum Component Distance (MCD) GIRGs}
In \cref{sec:GIRG_def} we introduced the max norm $\norm{\cdot}_\infty$, euclidean norm, and the minimum component distance (MCD) $\norm{\cdot}_{\mathrm{mcd}}$ as alternative distance functions.
While the max norm is commonly used in e.g. \cite{bringmann2019geometric} as its handy for proofs, the MCD can make more sense in some settings. For instance in social networks, the max norm is like stipulating that people only make friends with others that \q{tick all the boxes}, i.e. are similar in every dimension (this sounds more like finding a romantic partner). The MCD is saying that you make friends with people who are similar in at least one dimension. E.g. an individual might play in the local football club, and sing in the local choir - each a social circle formed from one shared hobby.

The MCD can be mixed with the max norm in an \q{and/or} fashion, e.g.
\begin{equation}
    \norm{x - y}_\text{mix} = \min \left ( ||\vec{x}_1 - \vec{y}_{1}||_\infty, ||\vec{x}_{[2,3]} - \vec{y}_{[2,3]}||_\infty, ||\vec{x}_{[4,5]} - \vec{y}_{[4,5]}||_\infty \right )
\end{equation}
This parses as points $x, y$ being \q{close} if they are close in the 1st dim, or the 2nd and 3rd dim, or the 4th and 5th dim. We call this a min/max mixed GIRG.


\subsection{Volume Formula for MCD, and Mixed Norms}
The formula for $Vol(r)$ differs by the distance function used $r(\vec{x}, \vec{y})$. For the max norm in $d$ dimensions we saw that $Vol(r) = (2r)^d$ as a cube with side-length $2r$. For the euclidean norm this becomes $Vol(r) = \frac{\pi^{d/2}}{\Gamma(d/2 + 1)} r^d$ as a ball of radius $r$ - e.g. $2r;\; \pi r^2;\;, \frac{4}{3} \pi r^3$ for $d=1, 2, 3$ respectively.

For the MCD, we can derive a formula using independent events. Consider the ball $B_r(\vec{x})$ and $\vec{y} \sim U(\T^d)$ sampled uniformly from the torus. Let event $A_i$ be that $|y_i - x_i| < r$. Then $Vol(r) = P(\bigcup_{i=1}^d A_i)$. Since events $A_i$ are independent by orthogonality of the individual dimensions,
we have that
\begin{align*}
    Vol(r) & = P \left (\bigcup_{i=1}^d A_i \right ) 
    = 1 - P \left ( \bigcap_{i=1}^d \overline{A}_i \right )
    \\
    &= 1 - \prod_{i=1}^d P(\overline{A}_i) = 1 - (1 - 2r)^d
\end{align*}
This can be extended to min/max mixed GIRGs with $(I_i)_{i=1}^k$ a partition of $\{1, ..., d\}$
\begin{align*}
    r(\vec{x}, \vec{y}) &= \min \left ( \norm{\vec{x}_{I_1} - \vec{y}_{I_1}}_\infty, ..., \norm{\vec{x}_{I_k} - \vec{y}_{I_k}}_\infty  \right )
    \\
    \text{then}\; Vol(r) &= 1 -  \left [ \prod_{i=1}^k 1 - (2r)^{|I_i|} \right ]
\end{align*}

% Then $Vol(r) = \PP(\bigcap_i A_i) = \prod_i \PP(A_i) = \prod_i \frac{1}{2r} = \frac{1}{(2r)^d}$.




% Finally the average degree formula in \cref{eq:expected_edge_degree} is just averaging over all terms in \cref{eq:p_u_to_v_marginal_on_position} for each pair of vertices $u, v$.


% The more general formula replaces replaces $||x_u - x_v||^d = r_{uv}^d$ with $Vol(B_r)$ the volume of the ball of radius $r=r_{uv}$.
% For norms, $Vol(B_r) = \Theta(r^d)$, e.g. in the $\infty$-norm, $Vol(B_r) = (2r)^d$ as a cube with side-length $2r$.
% The euclidean ball has some $\pi$'s in the formula. For the minimum component distance, $Vol(B_r) = \Theta(r)$.
% Volumetric equivalence makes sense in that the edge probability $p_{uv} | x_u, w_u, w_v$ in the Torus is determined by integrating $\int_{r=0}^{r=1} p_{uv}(r) p(r) dr = \int_{Vol=0}^{Vol=1} p_{uv}(Vol) dVol$, where $Vol(r) = Vol(B_r)$ is the volume of the ball of radius $r$.
% Hence across different GIRGs of different dimensions $d$, keeping $p_{uv}(Vol)$ the same function therefore keeps the pairwise edge probabilities $p_{uv}$ the same (but not the joint edge probability distribution of course).


\subsection{MCD GIRG speedy combination of 1d GIRGs generating algorithm}
We saw in \cref{sec:blasius_cpp_generation} that the max torus GIRG implementation of \cite{blasius2022efficiently} has a runtime of $O(n)$ for fixed $d$, but this scales exponentially in $d$. We propose a faster algorithm that basically generates an MCD GIRG by generating $d$ 1d GIRGs each in $O(n)$ time, and combining them together into one graph in total $O(dn)$ time. This gives a significant advantage to using MCD GIRGs over max GIRGs already for dimensions as low as $d=4$.

As an example, if we want to generate an MCD GIRG of dim $d=3$, it actually looks quite like a union of $3$ one dimensional GIRGs $G_1, G_2, G_3$, each sharing the same node weight sequence, but with independently 1d distributed node locations. Critically, the majority of edges in the resultant 3d MCD GIRG $G$ occur between nodes which happen to be quite close in just one dimension, and not in the other dimensions. This means that we don't have to significantly over generate many edges in the 1d GIRGs to produce our MCD GIRG.

If $\alpha=\infty$ the \q{union of 1d GIRGs} statement is almost true. For a node pair $(u,v)$, let $r_1, r_2, r_3$ be the distances in the $3$ 1d GIRGs and $p_1, p_2, p_3$ the corresponding edge probabilities. WLOG letting $r_1 \leq r_2 \leq r_3$, then $\alpha = \infty \implies p_i \in \{0, 1\}$, so in particular $u \stackrel{3}{\sim} v \implies u \stackrel{2}{\sim} v \implies u \stackrel{1}{\sim} v$. This means we could directly take the union over all edges in $G_1, G_2, G_3$ to produce our 3d MCD GIRG $G$. In the $\alpha < \infty$ case we rather must additionally check that any final edge does come from the smallest $r_i$ graph. E.g. it could be that $p_1 = 0.5, p_2 = 0.3, p_3 = 0.01$, and that $u \sim v$ only in $G_2$ - in this case we would take $u \not \sim v$ in $G$ as the minimum component is $i=1$, thus we copy edge/non-edge for $(u, v)$ from $G_1$. 

There's a subtle difference however in that the $3$ 1d GIRGs actually need edge probabilities corresponding to those in the resultant MCD GIRG. This means taking
\begin{align*}
    & p_i = \min \left \{
        1, c \left ( 
            \frac{w_u w_v / W}{1 - (1 - 2 r_i)^3}
        \right )     
    \right \}
\end{align*}
rather than the standard 1d GIRG edge probability which just has $2 r_i$ in the denominator. This pleasingly gives us that $p_i \leq p_{uv}$, since $r_{uv} := \min_j r_j \leq r_i$ where $p_{uv}$ and $r_{uv}$ are the edge probability, node pair distance respectively in the final 3d MCD GIRG. This means that in expectation the number of edges in each 1d GIRG is fewer than in the resultant MCD GIRG, i.e. $O(n)$ - this matches our \q{union of 1d GIRGs} intuition.

\paragraph{Extension to max GIRGs} We would like to use this higher dimension GIRG generating algorithm for max GIRGs as well. Unfortunately, although the algorithm is still correct, its runtime blows up due to the fact that each 1d GIRG will now have many more than $\Theta(n)$ edges. In short, now $r_{uv} := \max_j r_j \geq r_i$, so each 1d GIRG in expectation has more edges than the resultant max GIRG. 

% TODO the Blasius algorithm is "O(n)". It's definitely at least $O(|E|)$, but is it worse?

To generate the individual 1d GIRGs, we need at least complexity in their number of edges $O(|E|)$ which is now much larger than $\Theta(n)$. We give a sketchy intutition that can be made more rigorous:

For a node $u$ of fixed weight, consider the neighbours of $u$ of small weight $w_v$, connected to $u$ by a short edge, in both the resultant d-dim max GIRG, and one of the individual 1d GIRGs.
% , i.e. $p_{uv} = 1$ due to $c \left ( \frac{w_u w_v / W}{(2r)^d} \right ) \geq 1$
In both, these edges constitute on average a constant fraction of all of $u$'s edges.
In the resultant d-dim max GIRG, these come from a box of side length roughly $2r = \left ( \frac{\overline{\text{deg}}}{ n} \right )^{1/d}$ about $u$, resulting in $u$ having about $\overline{deg}$ neighbours.
In contrast, in the 1d GIRG of dimension $i$, the whole strip of nodes whose projection thereon is in the range $[(\vec{x}_u)_i - r, (\vec{x}_u)_i +r]$ will end up as neighbours of $u$.
This gives $u$ a much higher degree in the 1d GIRG, of at least $\widetilde{\text{deg}} = 2nr = \Theta(n^{1 - 1/d})$ in expectation.


% Focusing on the $\Theta(1)$ fraction of edges in the resultant max GIRG which are both short and between nodes of small weight, i.e. $p_{uv} = 1$ due to $c \left ( \frac{w_u w_v / W}{(2r)^d} \right ) \geq 1$ and e.g. $w_u, w_v \in [1, 2]$. Then for a fixed average degree $\overline{\text{deg}}$ of the max GIRG, we can roughly say that the neighbours of a node $u$ are exactly the nodes contained in a box of volume $\frac{\overline{\text{deg}}}{n}$ around it, i.e. one of side length $2r  = \left ( \frac{\overline{\text{deg}}}{ n} \right )^{1/d}$.

% In contrast, considering the 1d GIRG whose node locations are the projection of all nodes onto one of the dimensions $i$, the whole strip of $[0, 1]^d$ of volume $2r$ gets projected down into the segment $[(\vec{x}_u)_i - r, (\vec{x}_u)_i +r]$.
% This contains in expectation $n \cdot 2r$ nodes, and are all neighbours of $u$ in the ith 1d GIRG. Putting this together means that average degree in the 1d GIRG is $\widetilde{\text{deg}} = 2nr = \Theta(n^{1 - 1/d})$.

This means that generating a $d$-dimensionsal max GIRG with our combination of 1d GIRGs algorithm has runtime at least $O(d \cdot n^{2 - 1/d})$. This is still better than the naive $O(n^2)$ algorithm, and is actually faster than exponential in d, linear in n: $O(e^{kd} n)$, in the regime where $d \gg \log n$.


\paragraph{Extension to mixed GIRGs}
Although the algorithm doesn't have as good a runtime on max GIRGs, it's easy to see that it does extend well to a mixed GIRG whose distance function has an outer minimum - of the form $r(\vec{x}, \vec{y}) = \min \left ( \norm{\vec{x}_{I_1} - \vec{y}_{I_1}}_1, ..., \norm{\vec{x}_{I_k} - \vec{y}_{I_k}}_k  \right )$. We just need a routine to sample the GIRGs in each dimension subset which also returns the distance in that subset for each edge. Then we can combine each of these GIRGs' edges together to form the final output GIRG. The number of overcounted (and hence needed to be discarded) edges from the subset GIRGs won't exceed the final edge count by more than a factor of $d$. This depends on the point locations - e.g. in the simple $d=2$ MCD GIRG case we expect little overcounting, however if the points all lie close to a linear subspace $x_1 = x_2$, then we get almost exactly double counting.

Therefore we can efficiently sample e.g. high dimensional mixed min/max GIRGs as long as each inner max dimension subset is small, such as $|I_j| \leq 3$.



\subsection{Distorted GIRGs}
\label{subsec:distorted_girgs}
A final GIRG variant that we only play with in \cref{chap:diff_maps} is the \textit{distorted GIRG}. This is a GIRG where the node locations' dimensions aren't all treated equally - rather they are ordered in importance $1 > 2 > ... > d$ via some weighting scheme $1=\lambda_1 \geq \lambda_2 \geq ... \lambda_d > 0$. For instance a max norm distorted torus GIRG would have $r(\vec{x}, \vec{y}) = \max_i \lambda_i |x_i - y_i|_C$. Oddly, to distort a MCD GIRG you instead have to use the inverse weights $\lambda_i^{-1}$, as now the smallest dimension distance has the greatest influence on the overall node distance.

The idea of distorted GIRGs is natural - that some dimension are more important than others in influencing edges. For social networks it could be e.g. that socioeconomic status of individuals has much higher impact on their friendships than whether they're into football or basketball. The weighted representation is easy to work with but an alternative is to literally distort the space $\chi = \T^d$ or $\chi = [0,1]^d$ to be cuboidal, i.e. no longer cube shaped.

% Our 1d modified to be $d$-d edge probability like GIRGs have strictly more edges than their normal 1d counterparts, as they have $(2r_i)^d \leq 2r_i$ in their $p_{uv}$ denominator.

% GIRGs have whp $\Theta(1)$ fraction of their nodes with "small" weight e.g. $w_u \in [1, 2]$, and also $\Theta(1)$ fraction of their edges as short: $p_{uv} = 1$ due to $c \left ( \frac{w_u w_v / W}{(2r)^d} ) \geq 1$. If we restrict to the subset of small weight nodes and consider their 

% Consider a short edge $u \sim v$ between small weight $u,v$ in the final max GIRG. As the edge is short, each of the $d$ 1d GIRGs is guaranteed to have an edge $u \sim v$ as well. Since we have restricted to small weights, the individual edge probabilities $P_i(u \sim v)$ of the ith 1d GIRG are quite independent.


% \begin{sidewaystable}
%     \centering
%     \begin{tabular}{|c|c|}
%         \hline
%         1-co & 1d copy-weight GIRG \\
%         1-23 & $1 \lor (2 \land 3)$ mixed min/max GIRG \\
%         2-min & $1 \lor 2$ 2d min GIRG \\
%     \end{tabular}
% \end{sidewaystable}


% Translating the two equivalent volumetric and distance based formulations of the Toroidal GIRG to the Cube actually gives two different but both potentially sensible definitions. 

% We'll start with the Toroidal Max GIRG where the distance function is $r^\T(x_u, x_v) = \norm{x_u - x_v}_{\infty, \T}$, the derived volume function in the toroidal setting is $Vol^\T(r^\T_{uv}) = (2r^\T_{uv})^d$, and the toroidal edge probability is $p_{uv}(r^\T) = \min \left \{1, \left ( \frac{w_u w_v}{n Vol^\T(r_{uv})} \right )^\alpha \right \} = \min \left \{1, \left ( \frac{w_u w_v}{n (2r^\T_{uv})^d} \right )^\alpha \right \}$. 

% The distance based cube GIRG connection probabilities just replaces $r^\T_{uv}$ with the cube distance $r_{uv}$, which is in general equal, or longer depending on the positions $x_u, x_v$.

% The translation of $Vol^\T(r^\T_{uv})$ into a cube geometry now no longer makes sense without a reference point about which the ball volume is measured. Hence it is replaced with $\sqrt{V_u(r_{uv}) V_v(r_{uv})}$, where $V_u(r_{uv})$ is the volume of the $r_{uv}$ ball about point $x_u$, only including intersection with the cube. This definition partially assuages the reduced connectivity of points $u$ that are near the extremes of the cube - their $V_u(r_{uv})$ is smaller than if they were more centrally located, and hence they strive more strongly to connect - however their potential neighbours $v$ will still usually be more centrally located and hence less willing to reciprocate the connection.

% The upshot connectivity comparison between the Torus and Cube GIRG, is that connection probabilities for centrally located nodes are generally the same. For distance based Cube GIRGs, connection probabilities everywhere are upper bounded by the Toroidal GIRG, and extremely located nodes have fewer neighbours. For volumetric Cube GIRGs, centrally located nodes actually have more neighbours than before, and extremely located nodes have fewer neighbours, but likely not in any balanced fashion.

% connection probability is higher. However it does not fully assuage this, as the volume of the $r_{uv}$ ball about $x_u$ is still smaller than if it were a torus, as the ball is cut off by the cube. 

% using now $p_{uv}(r) = \min \left \{1, \left ( \frac{w_u w_v}{n \sqrt{V_u(r_{uv}) V_v(r_uv)})} \right )^\alpha \right \}$, where now $r_{uv}$ is cube distance - so in general equal or longer than torus distance, and $V_u(r_{uv})$ is the volume of the $r_{uv}$ ball about point $u$, only including intersection with the cube.

% with just the fact that the underlying space is a Cube rather than a Torus changing the connection probabilities. The second is a more direct distance based formulation, $p_{uv}(r) = \min \left \{1, \left ( \frac{w_u w_v}{n r_{uv}^d} \right )^\alpha \right \}$. To be clear, these two formulations are for a given fixed distance function $r_{uv} = r(x_u, x_v) = ||x_u - x_v||_\infty$. 

% The volumetric formulation has the benefit that all nodes have the same expected degree profile no matter their location, whereas for the distance based formulation, a node near the edge of the cube would have fewer neighbours in expectation than if it were more centrally located. The distance based formulation 



\begin{comment}
\section{Weird GIRGs}
The generalisation formula is given by 
\begin{equation}
    p_{uv} = \min \left \{ 
        1,
        c \left (
            \frac{w_u w_v}{Vol(r_{uv})}
        \right )^\alpha    
    \right \}
\end{equation}
Where $r_{uv} = ||x_u - x_v||$. This is in the schema where $Vol(Torus) = n$. This is designed such that a $dV$ section of volume with approximately same radius $r$ from $x_u$ contains about $dV$ points. 


Ooooops
Ok

The whole point is that we want $p(u \sim v | x_u, w_u, w_v) = \Theta(w_u w_v)$ a la Chung-Lu.
$blah = \int p_{uv}(r) p(r) dr = V(\hat{r}) + \int_{\hat{r}}^{n^{1/d}} p_{uv}(r) p(r) dr$.
Both terms are Theta the same amount. $\hat{r}$ is defined to be the point when $p_{uv}(r) = 1$, i.e. $r = \Theta(w_u w_v)$. You can also see that the second integral fits this too. Since $p(r) = \dot(V)/n$, we get $\int ... p(r) dr = \int (w_u w_v / V)^\alpha dV = (w_u w_v)^\alpha \int 1/V^\alpha dV$. Now $\int_{\hat{V}}^n 1/V^\alpha dV = \hat{V}^{1 - \alpha} = (w_u w_v)^{1- \alpha}$.


This is kind of beautiful. It doesn't really matter what Volume function we use, as long as the volume of the whole Torus is fixed and makes sense (I guess either n or 1). This results in having exactly the same average degree, regardless of MAX GIRG or MIN GIRG or whatever. For the MAX GIRGs, Blasius actually calculated precisely $\E[X_{uv} | w_u w_v]$, via some volume/radius integrals, to get some complex formula s.t. $\E[\bar{d}] = f(c) = c^\alpha (...) + c(...)$, which is then numerically solved to find $\hat{c}: f(\hat{c}) = \text{desiredAvgDegree}$. We can then precisely use this same constant $c$ in our MIN GIRG shenanigans:


MCD GIRGs vs MAX GIRGs average degree.
As far as I can tell, they should have the same avg degree, and this is borne out. Note PointTorus uses $Vol(r) = r^d$ whereas PointTorus2 uses $Vol(r) = (2r)^d$ which is actually the correct volume, and hence a scale factor is used between the two.

Unfortunately this geometric space equivalence only holds on all toroidal geometries, so won't work for CUBE GIRGs I believe.

\begin{verbatim}
    n = 1500
    d = 3
    tau = 2.1
    alpha=1.2
    desiredAvgDegree=100.0
    
    g, edges, weights, pts_torus, const = generation.generate_GIRG_nk(n, d, tau, alpha, desiredAvgDegree=desiredAvgDegree, points_type=points.PointsTorus)
    print(const)
    utils.avg_degree(g)
    print()
    
    g, edges, weights, pts_torus2, _ = generation.generate_GIRG_nk(n, d, tau, alpha, points_type=points.PointsTorus2, weights=weights, const=const*(2**d))
    print(_)
    utils.avg_degree(g)
    print()
    
    g, edges, weights, pts_torus2, _ = generation.generate_GIRG_nk(n, d, tau, alpha, points_type=points.PointsMCD, weights=weights, const=const*(2**d))
    print(_)
    utils.avg_degree(g)

    =============================================

    1.3638471818782616
    100.10933333333334

    10.910777455026093
    100.004

    10.910777455026093
    100.44133333333333

\end{verbatim}

\chapter{GIRG fitting}


\chapter{Classification :(}


% \paragraph{Example Paragraph}
% \subparagraph{Example Subparagraph}
\end{comment}