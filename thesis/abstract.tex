\begin{abstract}
  Real world graphs, in particular social networks, exhibit clustering and are proposed to have an underlying geometrical generative structure.
  The GIRG generative graph model is one such proposed geometric model. We show good capability of different GIRG variants to produce graphs resembling a dataset of real social network graphs in a range of high level graph features.
  We then explore practical methods of fitting GIRG node geometric location parameters to an input graph. Using this we show that low dimensional GIRGs are capable of a good amount of replication on a node and edge level of our social network graphs. Throughout we hope to provide guidance on methods in how to generate and fit different types of GIRG, as well as an intuitive understanding of the GIRG model.
\end{abstract}
