\begin{abstract}
  Many real-world graphs, in particular social networks, exhibit clustering and are hypothesised to possess an underlying geometrical generative structure.
  The Geometric Inhomogeneous Random Graphs (GIRG) model is one proposed geometric generative model. Our main contribution is to demonstrate the proficiency of assorted GIRG variants in generating graphs that resemble a dataset of real social network graphs across a spectrum of high-level graph features.
  We then investigate practical methods of fitting GIRG node geometric location parameters to a given input graph, and show that as low as one dimensional GIRGs can successfully replicate our social network graphs on a node and edge level. Throughout the thesis, we aim to offer guidance on generating and fitting different types of GIRGs, as well as an intuitive understanding of the GIRG model.
\end{abstract}
